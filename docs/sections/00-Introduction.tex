\section*{Introduction}
\addcontentsline{toc}{section}{Introduction} % Adds this section to the table of contents
The initial distribution of stellar masses as the outcome of star formation is a fundamentally important key for understanding the evolution of stellar systems on star-cluster and
galaxy scales. \cite{yan-kroupa}. 
\\The stellar initial mass function $(IMF, \xi(m))$ describes the distribution of masses of stars, whereby $dN = \xi(m)dm$ is the number of stars formed in the mass interval $m, m + dm$. It is one of the most important distribution functions in astrophysics as stellar evolution is generally determined by the mass of the stars. The IMF therefore regulates the chemical enrichment history of galaxies, as well as their mass-to-light ratios and influences their dynamical evolution.\cite{kroupa-weidner} \\
Theoretically unexpected, the IMF is found to be invariant through a large range of conditions like gas densities and metallicites (Kroupa 2001, 2002;
Chabrier 2003; Elmegreen et al. 2008; Bastian et al. 2010; Kroupa et al. 2013) and is well described by the canonical IMF (Appendix B).\\ Though, it has to be kept in mind that often the concept of an universal IMF is understood as a constant slope
of the IMF, ignoring the upper and lower mass limits. As the slope (for stellar masses above $1 M_{\odot}$ ) has been found to be
constant (within the uncertainties) for star clusters in the Milky Way and the Magellanic clouds (Kroupa 2002; Massey 2003),
an invariant IMF is widely used to not only describe individual star clusters but also stellar populations of whole galaxies.\\ But,
the question remains whether the IMF, derived from and tested on star cluster scales, is the appropriate stellar distribution
function for complex stellar populations like galaxies. \\ In this context, it has emerged that if all the stars in a galaxy form
with a canonical IMF 1 and all these IMFs of all star-forming events (spatially and temporally correlated star formation
events/CSFE) are added up the resulting integrated galactic initial mass function of stars (IGIMF) differs substantially from the canonical IMF. It should be pointed out here that the principal concept of the IGIMF - the galaxy-wide IMF (= IGIMF)
of a galaxy is always the sum of all star-formation events within a galaxy - is in any case always true.
\\
The ingredients for the IGIMF as applied here are listed as follows:
\begin{itemize}
	\item The IMF, $\xi(m)$, within star clusters is assumed to be canonical (see Appendix B),
	\item the CSFEs populate an embedded-cluster mass function (ECMF), which is assumed to be a power-law of the form,\\
		$\xi_{ecl}(M_{ecl}) = dN/dM_{ecl} \propto M_{ecl}^{-\beta}$
	\item the relation between the most-massive star in a cluster, $m_{max}$ , and the stellar mass of the embedded cluster, $M_{ecl}$
	(Weidner Kroupa 2004, 2006; Weidner et al. 2010),
	\item the relation between the star-formation rate (SFR) of a galaxy and the most-massive young (< 10 Myr) star cluster,
	$log_{10} (M_{ecl,max} ) = 0.746 \times log_{10}(SF R) + 4.93$ (Weidner et al. 2004).
	
\end{itemize}
